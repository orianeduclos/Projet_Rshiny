% Options for packages loaded elsewhere
\PassOptionsToPackage{unicode}{hyperref}
\PassOptionsToPackage{hyphens}{url}
%
\documentclass[
]{article}
\usepackage{amsmath,amssymb}
\usepackage[]{mathpazo}
\usepackage{iftex}
\ifPDFTeX
  \usepackage[T1]{fontenc}
  \usepackage[utf8]{inputenc}
  \usepackage{textcomp} % provide euro and other symbols
\else % if luatex or xetex
  \usepackage{unicode-math}
  \defaultfontfeatures{Scale=MatchLowercase}
  \defaultfontfeatures[\rmfamily]{Ligatures=TeX,Scale=1}
\fi
% Use upquote if available, for straight quotes in verbatim environments
\IfFileExists{upquote.sty}{\usepackage{upquote}}{}
\IfFileExists{microtype.sty}{% use microtype if available
  \usepackage[]{microtype}
  \UseMicrotypeSet[protrusion]{basicmath} % disable protrusion for tt fonts
}{}
\makeatletter
\@ifundefined{KOMAClassName}{% if non-KOMA class
  \IfFileExists{parskip.sty}{%
    \usepackage{parskip}
  }{% else
    \setlength{\parindent}{0pt}
    \setlength{\parskip}{6pt plus 2pt minus 1pt}}
}{% if KOMA class
  \KOMAoptions{parskip=half}}
\makeatother
\usepackage{xcolor}
\usepackage[left=2.5cm,right=2.5cm,top=2cm,bottom=2cm]{geometry}
\usepackage{graphicx}
\makeatletter
\def\maxwidth{\ifdim\Gin@nat@width>\linewidth\linewidth\else\Gin@nat@width\fi}
\def\maxheight{\ifdim\Gin@nat@height>\textheight\textheight\else\Gin@nat@height\fi}
\makeatother
% Scale images if necessary, so that they will not overflow the page
% margins by default, and it is still possible to overwrite the defaults
% using explicit options in \includegraphics[width, height, ...]{}
\setkeys{Gin}{width=\maxwidth,height=\maxheight,keepaspectratio}
% Set default figure placement to htbp
\makeatletter
\def\fps@figure{htbp}
\makeatother
\setlength{\emergencystretch}{3em} % prevent overfull lines
\providecommand{\tightlist}{%
  \setlength{\itemsep}{0pt}\setlength{\parskip}{0pt}}
\setcounter{secnumdepth}{5}
\ifLuaTeX
  \usepackage{selnolig}  % disable illegal ligatures
\fi
\usepackage[]{natbib}
\bibliographystyle{plainnat}
\IfFileExists{bookmark.sty}{\usepackage{bookmark}}{\usepackage{hyperref}}
\IfFileExists{xurl.sty}{\usepackage{xurl}}{} % add URL line breaks if available
\urlstyle{same} % disable monospaced font for URLs
\hypersetup{
  pdftitle={Les Naissances},
  hidelinks,
  pdfcreator={LaTeX via pandoc}}

\title{Les Naissances}
\usepackage{etoolbox}
\makeatletter
\providecommand{\subtitle}[1]{% add subtitle to \maketitle
  \apptocmd{\@title}{\par {\large #1 \par}}{}{}
}
\makeatother
\subtitle{Projet de visualisation de données\\
R Shiny\\
Université de Rennes II : Master Mathématiques Appliquées, Statistiques}
\author{Margaux Bailleul\\
Oriane Duclos}
\date{11 March, 2023}

\begin{document}
\maketitle

{
\setcounter{tocdepth}{2}
\tableofcontents
}
Lien de l'application :

Lien du github :

\hypertarget{introduction}{%
\section{Introduction}\label{introduction}}

\hypertarget{pourquoi-avons-nous-choisi-de-travailler-sur-les-naissances}{%
\subsection{Pourquoi avons-nous choisi de travailler sur les naissances
?}\label{pourquoi-avons-nous-choisi-de-travailler-sur-les-naissances}}

Les naissances est un sujet très intéressant et qui peut présenter
énormément de possibilités de traitements visuels : cartes, régression
linéaire, graphiques\ldots{}

\hypertarget{uxe0-quoi-sert-notre-application}{%
\subsection{À quoi sert notre application
?}\label{uxe0-quoi-sert-notre-application}}

Notre application se veut tout d'abord intercative : on veut que
l'utilisateur puisse avoir les informations qu'il veut quand il le
souhaite Elle se veut également simple d'utilisation Nous voulions
traiter les naissances à différentes échelles pour rendre compte de la
différence

\hypertarget{pruxe9sentation-des-diffuxe9rentes-bases-de-donnuxe9es}{%
\subsection{Présentation des différentes bases de
données}\label{pruxe9sentation-des-diffuxe9rentes-bases-de-donnuxe9es}}

\hypertarget{base-de-donnuxe9es-wdi}{%
\subsubsection{Base de données WDI}\label{base-de-donnuxe9es-wdi}}

\hypertarget{base-de-donnuxe9es-etat_civil}{%
\subsubsection{Base de données
etat\_civil}\label{base-de-donnuxe9es-etat_civil}}

\hypertarget{base-de-donnuxe9es-bebe}{%
\subsubsection{Base de données bebe}\label{base-de-donnuxe9es-bebe}}

Il s'agit d'une base de donnée que nous avons utilisé dans le cadre du
cours de Logiciel avancé avec Nicolas Jegou. Celle ci étoffe notre étude
en completant notre base de données etat\_civil. Nous avons de nouvelle
variable pertinantes en ce qui concerne le poids, la taille de la mère
et du bébé par exemple. Variable que nous n'avons pas dans la base de
donnée etat\_civil.

\textbf{Tri de bebe} : Valeur manquante Nous avons decider de faire un
tri sur les individus non totalement renseigné et d'utiliser tableau
sans NA

\hypertarget{base-de-donnuxe9es-taux_fecondituxe9}{%
\subsubsection{Base de données
taux\_fecondité}\label{base-de-donnuxe9es-taux_fecondituxe9}}

La base de données taux de fécondité

\hypertarget{etude-des-naissances}{%
\section{Etude des naissances}\label{etude-des-naissances}}

\hypertarget{regression}{%
\subsection{Regression}\label{regression}}

Nous allons etudier le poids de naissance des bébés.

Les variables sont :

\begin{itemize}
\item
  le poids de naissance du bébé (en grammes) (\textbf{PoidsBB})
\item
  l'âge de la mère (\textbf{AgedeleMere})
\item
  le poids de la mère en ??? (\textbf{PoidsMere})
\item
  la taille du bébé (en centimètre) (\textbf{TailleBB})
\item
  Sexe du bébé (fille = 0, garçon = 1) : transformation de la variable
  \textbf{Sexe} en varaible indicatrice
\end{itemize}

Nous considérons le modèle suivant :

\(Y_{PoidsBB} = \beta0 + \beta1 X1_{AgedelaMere} + \beta2 X2_{PoidsMere} + \beta3X3_{TailleBB} + \beta4X4_{Sexe} + \epsilon\)

Avant d'estimer les paramêtres, nous calculons la matrice de corrélation
et nous présenterons un diagramme de dispersion de toutes les pairs de
ces varaibles. Ceci permet de visualiser la relation entre la variable à
expliquer et chacune des variables explicatives et de juger de la
correlation entre les variables explicatives

\hypertarget{worldcoud}{%
\subsection{Worldcoud}\label{worldcoud}}

Le nuage de mots ou « wordcloud » en anglais est un outil de
visualisation qui permet au travers d'une image de percevoir très
rapidement quels sont les mots qui sont les plus fréquents au sein d'un
texte ou un corpus de texte. L'utilisateur peut en un clic selectionner
une année et observer quels sont les prénoms qui ont été le plus
fréquament donnée sur cette année. Qu'est ce que son prénom à été
fréquament donnée à son année de naissance ?

Wordcloud2est un widget html. Cela signifie que votre wordcloud sera
sorti dans un HTMLformat. Nous avons décider de mettre en place un
bouton qui permet à l'utilisateur de l'exporter en tant que png image.

\hypertarget{travail-en-binuxf4me}{%
\section{Travail en binôme}\label{travail-en-binuxf4me}}

\hypertarget{utilisation-de-git}{%
\subsection{Utilisation de git}\label{utilisation-de-git}}

\hypertarget{ruxe9partition-des-tuxe2ches}{%
\subsection{Répartition des tâches}\label{ruxe9partition-des-tuxe2ches}}

\hypertarget{notre-ressenti}{%
\subsection{Notre ressenti}\label{notre-ressenti}}

\hypertarget{difficultuxe9s-rencontruxe9es}{%
\subsection{Difficultés
rencontrées}\label{difficultuxe9s-rencontruxe9es}}

Margaux : utilisation du package formattable, association avec le
package DT qui marchait très bien dans ma console mais impossible de le
faire marcher dans l'application à cause d'une erreur html Base de
données fertility, impossible de sélectionner trop d'années car trop
lourd NA dans la carte, on ne sait pas comment l'enlever Voulait faire
un bouton logout mais il fallait shiny server pro

\hypertarget{conlusion}{%
\section{Conlusion}\label{conlusion}}

Pour conclure, cela a été un vrai plaisir de travailler sur ce projet.

\end{document}
