% Options for packages loaded elsewhere
\PassOptionsToPackage{unicode}{hyperref}
\PassOptionsToPackage{hyphens}{url}
%
\documentclass[
]{article}
\usepackage{amsmath,amssymb}
\usepackage[]{mathpazo}
\usepackage{iftex}
\ifPDFTeX
  \usepackage[T1]{fontenc}
  \usepackage[utf8]{inputenc}
  \usepackage{textcomp} % provide euro and other symbols
\else % if luatex or xetex
  \usepackage{unicode-math}
  \defaultfontfeatures{Scale=MatchLowercase}
  \defaultfontfeatures[\rmfamily]{Ligatures=TeX,Scale=1}
\fi
% Use upquote if available, for straight quotes in verbatim environments
\IfFileExists{upquote.sty}{\usepackage{upquote}}{}
\IfFileExists{microtype.sty}{% use microtype if available
  \usepackage[]{microtype}
  \UseMicrotypeSet[protrusion]{basicmath} % disable protrusion for tt fonts
}{}
\makeatletter
\@ifundefined{KOMAClassName}{% if non-KOMA class
  \IfFileExists{parskip.sty}{%
    \usepackage{parskip}
  }{% else
    \setlength{\parindent}{0pt}
    \setlength{\parskip}{6pt plus 2pt minus 1pt}}
}{% if KOMA class
  \KOMAoptions{parskip=half}}
\makeatother
\usepackage{xcolor}
\usepackage[left=2.5cm,right=2.5cm,top=2cm,bottom=2cm]{geometry}
\usepackage{color}
\usepackage{fancyvrb}
\newcommand{\VerbBar}{|}
\newcommand{\VERB}{\Verb[commandchars=\\\{\}]}
\DefineVerbatimEnvironment{Highlighting}{Verbatim}{commandchars=\\\{\}}
% Add ',fontsize=\small' for more characters per line
\newenvironment{Shaded}{}{}
\newcommand{\AlertTok}[1]{\textcolor[rgb]{1.00,0.00,0.00}{#1}}
\newcommand{\AnnotationTok}[1]{\textcolor[rgb]{0.00,0.50,0.00}{#1}}
\newcommand{\AttributeTok}[1]{#1}
\newcommand{\BaseNTok}[1]{#1}
\newcommand{\BuiltInTok}[1]{#1}
\newcommand{\CharTok}[1]{\textcolor[rgb]{0.00,0.50,0.50}{#1}}
\newcommand{\CommentTok}[1]{\textcolor[rgb]{0.00,0.50,0.00}{#1}}
\newcommand{\CommentVarTok}[1]{\textcolor[rgb]{0.00,0.50,0.00}{#1}}
\newcommand{\ConstantTok}[1]{#1}
\newcommand{\ControlFlowTok}[1]{\textcolor[rgb]{0.00,0.00,1.00}{#1}}
\newcommand{\DataTypeTok}[1]{#1}
\newcommand{\DecValTok}[1]{#1}
\newcommand{\DocumentationTok}[1]{\textcolor[rgb]{0.00,0.50,0.00}{#1}}
\newcommand{\ErrorTok}[1]{\textcolor[rgb]{1.00,0.00,0.00}{\textbf{#1}}}
\newcommand{\ExtensionTok}[1]{#1}
\newcommand{\FloatTok}[1]{#1}
\newcommand{\FunctionTok}[1]{#1}
\newcommand{\ImportTok}[1]{#1}
\newcommand{\InformationTok}[1]{\textcolor[rgb]{0.00,0.50,0.00}{#1}}
\newcommand{\KeywordTok}[1]{\textcolor[rgb]{0.00,0.00,1.00}{#1}}
\newcommand{\NormalTok}[1]{#1}
\newcommand{\OperatorTok}[1]{#1}
\newcommand{\OtherTok}[1]{\textcolor[rgb]{1.00,0.25,0.00}{#1}}
\newcommand{\PreprocessorTok}[1]{\textcolor[rgb]{1.00,0.25,0.00}{#1}}
\newcommand{\RegionMarkerTok}[1]{#1}
\newcommand{\SpecialCharTok}[1]{\textcolor[rgb]{0.00,0.50,0.50}{#1}}
\newcommand{\SpecialStringTok}[1]{\textcolor[rgb]{0.00,0.50,0.50}{#1}}
\newcommand{\StringTok}[1]{\textcolor[rgb]{0.00,0.50,0.50}{#1}}
\newcommand{\VariableTok}[1]{#1}
\newcommand{\VerbatimStringTok}[1]{\textcolor[rgb]{0.00,0.50,0.50}{#1}}
\newcommand{\WarningTok}[1]{\textcolor[rgb]{0.00,0.50,0.00}{\textbf{#1}}}
\usepackage{graphicx}
\makeatletter
\def\maxwidth{\ifdim\Gin@nat@width>\linewidth\linewidth\else\Gin@nat@width\fi}
\def\maxheight{\ifdim\Gin@nat@height>\textheight\textheight\else\Gin@nat@height\fi}
\makeatother
% Scale images if necessary, so that they will not overflow the page
% margins by default, and it is still possible to overwrite the defaults
% using explicit options in \includegraphics[width, height, ...]{}
\setkeys{Gin}{width=\maxwidth,height=\maxheight,keepaspectratio}
% Set default figure placement to htbp
\makeatletter
\def\fps@figure{htbp}
\makeatother
\setlength{\emergencystretch}{3em} % prevent overfull lines
\providecommand{\tightlist}{%
  \setlength{\itemsep}{0pt}\setlength{\parskip}{0pt}}
\setcounter{secnumdepth}{5}
\ifLuaTeX
  \usepackage{selnolig}  % disable illegal ligatures
\fi
\usepackage[]{natbib}
\bibliographystyle{plainnat}
\IfFileExists{bookmark.sty}{\usepackage{bookmark}}{\usepackage{hyperref}}
\IfFileExists{xurl.sty}{\usepackage{xurl}}{} % add URL line breaks if available
\urlstyle{same} % disable monospaced font for URLs
\hypersetup{
  pdftitle={Les Naissances},
  hidelinks,
  pdfcreator={LaTeX via pandoc}}

\title{Les Naissances}
\usepackage{etoolbox}
\makeatletter
\providecommand{\subtitle}[1]{% add subtitle to \maketitle
  \apptocmd{\@title}{\par {\large #1 \par}}{}{}
}
\makeatother
\subtitle{Projet de visualisation de données\\
R Shiny\\
Le contexte, les données, ce à quoi sert notre appli\\
Université de Rennes II : Master Mathématiques Appliquées, Statistiques}
\author{Margaux Bailleul\\
Oriane Duclos}
\date{19 mars, 2023}

\begin{document}
\maketitle

{
\setcounter{tocdepth}{2}
\tableofcontents
}
\begin{center}\rule{0.5\linewidth}{0.5pt}\end{center}

Lien de l'application :
\url{https://oxsb16-oriane-duclos.shinyapps.io/BAILLEUL_DUCLOS/}\\
Lien du github : \url{https://github.com/orianeduclos/Projet_Rshiny.git}

\hypertarget{introduction}{%
\section{Introduction}\label{introduction}}

\hypertarget{pourquoi-avons-nous-choisi-de-travailler-sur-les-naissances}{%
\subsection{Pourquoi avons-nous choisi de travailler sur les naissances
?}\label{pourquoi-avons-nous-choisi-de-travailler-sur-les-naissances}}

Les données sur les naissances permettent de comprendre les
comportements de reproduction des populations et de planifier les
services de santé en conséquence. Elles sont importantes pour les études
démographiques, telles que la projection de la croissance de la
population et la compréhension de la répartition géographique des
naissances. De plus, ces données sont utilisées dans la recherche en
santé publique pour mieux comprendre les facteurs qui influencent la
santé maternelle et infantile, et pour développer de nouvelles
interventions pour améliorer la santé des mères et des enfants.

Cela en fait donc un sujet très intéressant et qui peut présenter
énormément de possibilités de traitements visuels : cartes, régression
linéaire, graphiques\ldots{}

\hypertarget{uxe0-quoi-sert-notre-application}{%
\subsection{À quoi sert notre application
?}\label{uxe0-quoi-sert-notre-application}}

Notre application veut rendre la compréhension des naissances dans le
monde de manière accessible et à travers différents échelles. C'est pour
cela que nous avons rendu l'application intercative au maximum, en nous
permettant de visualiser et d'analyser des données. Cela va être rendu
possible grâce à l'utilisation de RShiny, qui permet de créer des
tableaux de bord interactifs, des graphiques interactifs et des
visualisations de données dynamiques, permettant aux utilisateurs de
filtrer, trier, explorer et analyser les données en temps réel. Elle se
veut également simple d'utilisation avec différents onglets traitant
chacun d'une échelle différente. L'application veut répondre à la
problématique suivante : comment les naissances évoluent-elles à
différentes échelles ?

\hypertarget{pruxe9sentation-des-diffuxe9rentes-bases-de-donnuxe9es-et-de-leurs-usages}{%
\subsection{Présentation des différentes bases de données et de leurs
usages}\label{pruxe9sentation-des-diffuxe9rentes-bases-de-donnuxe9es-et-de-leurs-usages}}

\hypertarget{base-de-donnuxe9es-bebe}{%
\subsubsection{Base de données bebe}\label{base-de-donnuxe9es-bebe}}

Il s'agit d'une base de données que nous avons utilisé dans le cadre du
cours de ``Logiciel avancé'' enseigné par Nicolas Jegou durant notre
licence. La base présente plusieurs variables quantitatives, nous avons
donc relevé des variables pertinantes en ce qui concerne le poids, la
taille de la mère ou encore celle du bébé par exemple.

\textbf{Tri de la base de données} : Valeurs manquantes \newline Nous
avons decidé de faire un tri sur les individus non totalement renseignés
et d'utiliser une base de données sans NA.

\hypertarget{base-de-donnuxe9es-taux_fecondituxe9}{%
\subsubsection{Base de données
taux\_fecondité}\label{base-de-donnuxe9es-taux_fecondituxe9}}

La base de données taux de fécondité nous indique la localisation,
l'indicateur, la fréquence, le temps et la valeur du taux de fécondité
dans les différents pays du monde en fonction des années (de 1960 à
2020).

\hypertarget{base-de-donnuxe9es-wdi}{%
\subsubsection{Base de données WDI}\label{base-de-donnuxe9es-wdi}}

La bibliothèque WDI (World Development Indicators) est un package R qui
permet de télécharger et d'explorer les indicateurs de développement
économique et social du monde entier. Cette bibliothèque est basée sur
la base de données de la Banque mondiale, qui comprend une grande
quantité de données. Nous utilisons cette librairie pour pouvoir montrer
à l'utilisateur le taux de fertilité dans le monde en fonction des
années. \newline De plus, en important la librairie, nous nous sommes
limitées aux années 2017, 2018 et 2019. Il s'agit d'une librairie qui
est lourde et qui met du temps à charger. Si nous avions choisi un plus
grand nombre d'années, cela aurait considérablement ralenti le
chargement de l'application.

\hypertarget{base-de-donnuxe9es-dpt2021}{%
\subsubsection{Base de données
dpt2021}\label{base-de-donnuxe9es-dpt2021}}

Cette base de données comporte les prénoms donnés à des bébés de 1900 à
2021 en France. Nous avons tout de suite vu la possibilité de faire de
la visualisation avec cette base, étant donné la grande période sur
laquelle elle s'étend, et étant en lien direct avec les naissances.

\hypertarget{etude-des-naissances}{%
\section{Etude des naissances}\label{etude-des-naissances}}

\hypertarget{uxe0-luxe9chelle-du-monde}{%
\subsection{À l'échelle du monde}\label{uxe0-luxe9chelle-du-monde}}

\hypertarget{carte}{%
\subsubsection{Carte}\label{carte}}

La carte nous permet de mettre en évidence les différents pays dont le
taux de fertilité est le plus élevé. Les seules années sélectionnables
sont de 2017 à 2019, car comme expliqué précédemment, augmenter la plage
de la période aurait considérablement ralenti notre application.

\hypertarget{taux-de-fuxe9condituxe9-des-pays-du-monde}{%
\subsubsection{Taux de fécondité des pays du
monde}\label{taux-de-fuxe9condituxe9-des-pays-du-monde}}

Le graphique présentant tous les pays est pertinent car il nous permet
de comparer directement les pays entre eux et d'avoir une idée de la
tendance globale de l'évolution du taux de fécondité au fur et à mesure
des années. Le graphique est intercatif et nous permet de savoir la
valeur du taux de fécondité pour chaque pays et pour chaque année.

\hypertarget{graphique-sur-un-seul-pays}{%
\subsubsection{Graphique sur un seul
pays}\label{graphique-sur-un-seul-pays}}

Le graphique présentant tous les pays n'est cependant pas suffisant. En
effet, nous avons du mal à avoir une idée précise de l'évolution d'un
seul pays. Cet onglet permet donc de rendre compte de l'évolution d'un
seul pays, en observant des hausses ou des baisses que nous ne pouvions
pas visualiser dans le graphique précédent.

\hypertarget{uxe0-luxe9chelle-de-la-france}{%
\subsection{À l'échelle de la
France}\label{uxe0-luxe9chelle-de-la-france}}

\hypertarget{wordcoud}{%
\subsubsection{Wordcoud}\label{wordcoud}}

Le nuage de mots ou « wordcloud » en anglais est un outil de
visualisation qui permet au travers d'une image de percevoir très
rapidement quels sont les mots qui sont les plus fréquents au sein d'un
texte ou un corpus de texte. L'utilisateur peut en un clic sélectionner
une année et observer quels sont les prénoms qui ont été le plus
fréquemment donnés sur cette année, en lui laissant le choix de la
fréquence d'apparition du prénom ainsi que le nombre de prénoms qui
seront présents dans le wordcloud.\\
Nous avons décidé de mettre en place un bouton qui permet à
l'utilisateur de l'exporter en tant que png image.

\hypertarget{courbe-du-pruxe9nom-au-fur-et-uxe0-mesure-des-annuxe9es}{%
\subsubsection{Courbe du prénom au fur et à mesure des
années}\label{courbe-du-pruxe9nom-au-fur-et-uxe0-mesure-des-annuxe9es}}

Cet onglet est sûrement celui qui amène le plus l'utilisateur à
interragir avec l'application. Nous sommes en effet amenés à écrire un
prénom, et en fonction de celui-ci, la courbe du nombre de bébés ayant
reçu ce prénom en fonction des années s'affichera. Il ne faut pas
hésiter à essayer les prénoms de personnes que nous connaissons.

Nous avons décider de mettre en place un bouton qui permet à
l'utilisateur de l'exporter en tant que png image.

\hypertarget{uxe0-luxe9chelle-dune-maternituxe9}{%
\subsection{À l'échelle d'une
maternité}\label{uxe0-luxe9chelle-dune-maternituxe9}}

\hypertarget{visualisation}{%
\subsubsection{Visualisation}\label{visualisation}}

Cet onglet nous permet de visualiser rapidement les statistiques
descriptives de la base de données bébé avec l'utilisation du package
RamCharts. Nous avons intégré différents types de répresentation
(boxplot, jauge..). Cela nous as permis d'utiliser un grand panel des
fonctionnalités de ce package.

Nous avons également mis en place un graphique représentant le poids du
bébé en fonction de la tranche d'âge de la maman.

\hypertarget{ruxe9gression-simple}{%
\subsubsection{Régression simple}\label{ruxe9gression-simple}}

Nous allons etudier le poids de naissance des bébés (en grammes).\\
Les variables explicatives sont toutes les variables quantitatives de
notre base de données.

Nous considérons le modèle suivant en fonction de ce que l'utilisateur
choisira:

\(Y_{PoidsBB} = \beta0 + \beta1 X1_{ChoixUtilisateur} + \epsilon\)

Nous représentons en clic bouton le choix des variables. L'utilisateur
peut ensuite visualiser la variable sélectionnée sous forme
d'histogramme, puis la visualiser avec la courbe de régression en
fonction de la variable PoidsBB.

\hypertarget{ruxe9gression-multiple}{%
\subsubsection{Régression multiple}\label{ruxe9gression-multiple}}

L'utilisateur a le choix de faire une régression multiple en choisissant
les différentes variables explicatives. Il pourra, dès que le choix est
fait, visualiser :

\begin{itemize}
\item
  La matrice de corrélation : la matrice de corrélation indique les
  valeurs de corrélation, qui mesurent le degré de relation linéaire
  entre chaque paire de variables. Les valeurs de corrélation peuvent
  être comprises entre -1 et +1. Si les deux variables ont tendance à
  augmenter et à diminuer en même temps, la valeur de corrélation est
  positive.
\item
  Le r² : le coefficient de détermination est un indicateur utilisé en
  statistiques pour juger de la qualité d'une régression linéaire. Ici,
  l'utilisateur pour voir avec un code couleur la qualité du modele.
  Plus la régression linéaire est en adéquation avec les données
  collectées.

  \begin{itemize}
  \tightlist
  \item
    Vert : R² supperieur ou égale à 0.6.
  \item
    Orange : R² est compris entre 0.4 et 0.6\\
  \item
    Rouge : R² est inferieur à 0.4
  \end{itemize}
\item
  La statistique de Fisher du modèle
\item
  Le summary du modèle de régression en sortie R
\end{itemize}

\hypertarget{travail-en-binuxf4me}{%
\section{Travail en binôme}\label{travail-en-binuxf4me}}

\hypertarget{utilisation-de-git}{%
\subsection{Utilisation de git}\label{utilisation-de-git}}

Afin de faciliter notre travail nous avons décidé d'utiliser git. En
effet, nous avons eu un cours en début de semestre et cela nous a semblé
pertinent de travailler avec cet outil. Nous avons pour ce projet
utilisé le terminal et non les boutons de R.

\hypertarget{ruxe9partition-des-tuxe2ches}{%
\subsection{Répartition des tâches}\label{ruxe9partition-des-tuxe2ches}}

Nous nous sommes réparties les tâches au fur et à mesure de l'avancé de
l'application. Nous avons tout d'abord décidé de la structure de
l'application ensemble, ainsi que les onglets et les sous-onglets qui
seront présents. Nous avons ensuite établi les grandes idées : les
graphiques, où nous voulions faire des cartes, quels packages nous
pouvions utiliser qui pouvaient rendre l'application intéressante etc.
Après cela, nous avons travailé en nous tenant au courant au fur et à
mesure pour continuer la répartition des tâches mais sur des tâches
beaucoup plus précises. Bien sûr, en cas de difficulté, l'entraide était
présente.

\hypertarget{notre-ressenti}{%
\subsection{Notre ressenti}\label{notre-ressenti}}

Nous avons beaucoup aimé travailler sur le thème des naissances. Avoir
différentes échelles à étudier nous a permis de ne pas nous répéter,
d'autant plus que nous avons utiliser plusieurs outils de visualisation
différents. Nous avons également beaucoup aimé travailler ensemble :
nous sommes un binôme qui fonctionne bien car nous avons la même méthode
de travail.

\hypertarget{difficultuxe9s-rencontruxe9es}{%
\subsection{Difficultés
rencontrées}\label{difficultuxe9s-rencontruxe9es}}

Nous avons chacune rencontré des difficultés en travaillant sur le
projet. Nous allons ici vous les présenter :

\begin{enumerate}
\def\labelenumi{\arabic{enumi}.}
\tightlist
\item
  Utilisation du package formattable, association avec le package DT qui
  marchait très bien dans ma console mais impossible de le faire marcher
  dans l'application à cause d'une erreur html
\item
  Base de données fertility, impossible de sélectionner trop d'années
  car trop lourd
\item
  NA dans le choix des années avec la carte, on ne sait pas comment
  l'enlever
\item
  Nous voulions faire un bouton logout mais était pour shiny server pro
\item
  Difficultés à publier notre application :
\end{enumerate}

\textbf{Methode de resolution des problèmes} \newline Pour trouver les
erreurs nous avons essayé de publier une application plus basique et
nous avons rajouter le code au fur et à mesure. Nous avons mis du temps
à réaliser cette étape car il faut vérifier étape par étape que notre
application fonctionne en local et la publier ensuite. Nous avons
utiliser la fonction ci-dessous pour déterminer nos erreurs de façon
précise.

\begin{Shaded}
\begin{Highlighting}[]
\NormalTok{rsconnect}\SpecialCharTok{::}\FunctionTok{showLogs}\NormalTok{() }\CommentTok{\#fonction pour afficher les messages du journal d\textquotesingle{}une application déployée}
\end{Highlighting}
\end{Shaded}

\textbf{Problèmes non résolus}

\begin{itemize}
\tightlist
\item
  Problème rencontré au niveau de la visualisation de notre base de
  données taux de fecondité.
\item
  Problème rencontré lorsque que l'on fait appel à un serveur externe
  type API qui nous permet de réaliser une carte du taux de fertilité
  avec le package WDI
\end{itemize}

\textbf{Les hypothèses du problème}

\begin{itemize}
\tightlist
\item
  Version shiny.io n'a pas les memes versions de packages que celles
  qu'on utilise en local
\item
  Shiny.io protege son serveur et ne peut pas publier des applications
  qui font appel à des serveurs externes
\end{itemize}

\textbf{Résolution de la publication}

Ces difficultés nous ont permis de voir les limites de notre application
et de nous remettre sans cesse en question. Concernant la publication de
l'application, nous avons donc trouvé les lignes de code qui nous
empêchaient de la publier. Nous avons alors mis en \# ces lignes, publié
l'application, et enlevé les \# afin que l'application ``entière''
puisse être visualisée si nous la lançons en local. Nous n'avons pas
publié notre carte qui fait appel au server externe WDI la table de
notre base de donnée taux de fécondité en affichage DT ainsi que son
summary. L'application publiée n'est donc pas complète, mais elle est
tout de même publiée.

\hypertarget{conlusion}{%
\section{Conlusion}\label{conlusion}}

Pour conclure, cela a été un vrai plaisir de travailler sur ce projet.
Nous espérons que l'application vous plaira. Nous avons beaucoup appris
: gérer nos erreurs, gérer notre temps (le projet demandant une certaine
investigation et ayant une date limite de rendu), répartir les
tâches\ldots{} Pour répondre à notre problématique, nous vous laissons
le loisir d'aller parcourir notre application.

\end{document}
