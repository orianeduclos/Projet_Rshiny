% Options for packages loaded elsewhere
\PassOptionsToPackage{unicode}{hyperref}
\PassOptionsToPackage{hyphens}{url}
%
\documentclass[
]{article}
\usepackage{amsmath,amssymb}
\usepackage[]{mathpazo}
\usepackage{iftex}
\ifPDFTeX
  \usepackage[T1]{fontenc}
  \usepackage[utf8]{inputenc}
  \usepackage{textcomp} % provide euro and other symbols
\else % if luatex or xetex
  \usepackage{unicode-math}
  \defaultfontfeatures{Scale=MatchLowercase}
  \defaultfontfeatures[\rmfamily]{Ligatures=TeX,Scale=1}
\fi
% Use upquote if available, for straight quotes in verbatim environments
\IfFileExists{upquote.sty}{\usepackage{upquote}}{}
\IfFileExists{microtype.sty}{% use microtype if available
  \usepackage[]{microtype}
  \UseMicrotypeSet[protrusion]{basicmath} % disable protrusion for tt fonts
}{}
\makeatletter
\@ifundefined{KOMAClassName}{% if non-KOMA class
  \IfFileExists{parskip.sty}{%
    \usepackage{parskip}
  }{% else
    \setlength{\parindent}{0pt}
    \setlength{\parskip}{6pt plus 2pt minus 1pt}}
}{% if KOMA class
  \KOMAoptions{parskip=half}}
\makeatother
\usepackage{xcolor}
\usepackage[left=2.5cm,right=2.5cm,top=2cm,bottom=2cm]{geometry}
\usepackage{graphicx}
\makeatletter
\def\maxwidth{\ifdim\Gin@nat@width>\linewidth\linewidth\else\Gin@nat@width\fi}
\def\maxheight{\ifdim\Gin@nat@height>\textheight\textheight\else\Gin@nat@height\fi}
\makeatother
% Scale images if necessary, so that they will not overflow the page
% margins by default, and it is still possible to overwrite the defaults
% using explicit options in \includegraphics[width, height, ...]{}
\setkeys{Gin}{width=\maxwidth,height=\maxheight,keepaspectratio}
% Set default figure placement to htbp
\makeatletter
\def\fps@figure{htbp}
\makeatother
\setlength{\emergencystretch}{3em} % prevent overfull lines
\providecommand{\tightlist}{%
  \setlength{\itemsep}{0pt}\setlength{\parskip}{0pt}}
\setcounter{secnumdepth}{5}
\ifLuaTeX
  \usepackage{selnolig}  % disable illegal ligatures
\fi
\usepackage[]{natbib}
\bibliographystyle{plainnat}
\IfFileExists{bookmark.sty}{\usepackage{bookmark}}{\usepackage{hyperref}}
\IfFileExists{xurl.sty}{\usepackage{xurl}}{} % add URL line breaks if available
\urlstyle{same} % disable monospaced font for URLs
\hypersetup{
  pdftitle={Les Naissances},
  hidelinks,
  pdfcreator={LaTeX via pandoc}}

\title{Les Naissances}
\usepackage{etoolbox}
\makeatletter
\providecommand{\subtitle}[1]{% add subtitle to \maketitle
  \apptocmd{\@title}{\par {\large #1 \par}}{}{}
}
\makeatother
\subtitle{Projet de visualisation de données\\
R Shiny\\
Université de Rennes II : Master Mathématiques Appliquées, Statistiques}
\author{Margaux Bailleul\\
Oriane Duclos}
\date{17 mars, 2023}

\begin{document}
\maketitle

{
\setcounter{tocdepth}{2}
\tableofcontents
}
le contexte, les données, ce à quoi sert votre appli

Lien de l'application :

Lien du github : \url{https://github.com/orianeduclos/Projet_Rshiny.git}

\hypertarget{introduction}{%
\section{Introduction}\label{introduction}}

\hypertarget{pourquoi-avons-nous-choisi-de-travailler-sur-les-naissances}{%
\subsection{Pourquoi avons-nous choisi de travailler sur les naissances
?}\label{pourquoi-avons-nous-choisi-de-travailler-sur-les-naissances}}

Les données sur les naissances permettent de comprendre les
comportements de reproduction des populations et de planifier les
services de santé en conséquence. Elles sont importantes pour les études
démographiques, telles que la projection de la croissance de la
population et la compréhension de la répartition géographique des
naissances. De plus, ces données sont utilisées dans la recherche en
santé publique pour mieux comprendre les facteurs qui influencent la
santé maternelle et infantile, et pour développer de nouvelles
interventions pour améliorer la santé des mères et des enfants.

Cela en fait donc un sujet très intéressant et qui peut présenter
énormément de possibilités de traitements visuels : cartes, régression
linéaire, graphiques\ldots{}

\hypertarget{uxe0-quoi-sert-notre-application}{%
\subsection{À quoi sert notre application
?}\label{uxe0-quoi-sert-notre-application}}

Notre application web se veut tout d'abord intercative. Celle ci nous
permet de visualiser et d'analyser des données à l'aide de R. Shiny
permet de créer des tableaux de bord interactifs, des graphiques
interactifs et des visualisations de données dynamiques, qui permettent
aux utilisateurs de filtrer, trier, explorer et analyser les données en
temps réel. Elle se veut également simple d'utilisation. Nous voulions
traiter les naissances à différentes échelles. En effet, à travers nos
différentes bases de données, les traitement varient énormément.

\hypertarget{pruxe9sentation-des-diffuxe9rentes-bases-de-donnuxe9es}{%
\subsection{Présentation des différentes bases de
données}\label{pruxe9sentation-des-diffuxe9rentes-bases-de-donnuxe9es}}

\hypertarget{base-de-donnuxe9es-wdi}{%
\subsubsection{Base de données WDI}\label{base-de-donnuxe9es-wdi}}

La bibliothèque WDI (World Development Indicators) est un package R qui
permet de télécharger et d'explorer les indicateurs de développement
économique et social du monde entier. Cette bibliothèque est basée sur
la base de données de la Banque mondiale, qui comprend une grande
quantité de données sur les pays du monde entier. Nous utilisons cette
library pour pouvoir montrer à l'utilisateur le taux de fertilité dans
le monde en fonction des années. De plus, nous avons que 2019, 2018 et
2017 en choix pour la variable année. Il s'agit d'une library qui est
lourde et met du temps à charger or, si nous laissons un plus garnd
choix d'année, cela ralentit considerablement le chargement de
l'application.

\hypertarget{base-de-donnuxe9es-bebe}{%
\subsubsection{Base de données bebe}\label{base-de-donnuxe9es-bebe}}

Il s'agit d'une base de donnée que nous avons utilisé dans le cadre du
cours de ``Logiciel avancé'' avec Nicolas Jegou. Nous avons des
variables pertinantes en ce qui concerne le poids, la taille de la mère
et du bébé par exemple.

\textbf{Tri de bebe} : Valeur manquante Nous avons decider de faire un
tri sur les individus non totalement renseignés et d'utiliser une base
de données sans NA.

\hypertarget{base-de-donnuxe9es-taux_fecondituxe9}{%
\subsubsection{Base de données
taux\_fecondité}\label{base-de-donnuxe9es-taux_fecondituxe9}}

La base de données taux de fécondité nous indique À COMPLETER

\hypertarget{base-de-donnuxe9es-dpt2021}{%
\subsubsection{Base de données
dpt2021}\label{base-de-donnuxe9es-dpt2021}}

Cette base de données comporte les prénoms donnés à des bébés de 1900 à
2014 en France. Nous avons tout de suite vu la possibilité de faire de
la visualisation avec cette base, étant donné la grande période sur
laquelle elle s'étend, et étant en lien direct avec les naissances.

\hypertarget{etude-des-naissances}{%
\section{Etude des naissances}\label{etude-des-naissances}}

\hypertarget{uxe0-luxe9chelle-du-monde}{%
\subsection{À l'échelle du monde}\label{uxe0-luxe9chelle-du-monde}}

\hypertarget{carte}{%
\subsubsection{Carte}\label{carte}}

La carte nous permet de mettre en évidence les différents pays dont le
taux de fertilité est le plus élevé. Sans surprise, c'est l'Afrique qui
se retrouve avec la plupart des pays avec un haut taux de fertilité.

\hypertarget{graphique-sur-tous-les-pays}{%
\subsubsection{Graphique sur tous les
pays}\label{graphique-sur-tous-les-pays}}

Le graphique présentant tous les pays est pertinent car il nous permet
de comparer directement les pays entre eux et d'avoir une idée de la
tendance globale de l'évolution du taux de fécondité au fur et à mesure
des années.

\hypertarget{graphique-sur-un-seul-pays}{%
\subsubsection{Graphique sur un seul
pays}\label{graphique-sur-un-seul-pays}}

Le graphique présentant tous les pays n'est cependant pas suffisant. En
effet, nous avons du mal à avoir une idée précise de l'évolution d'un
seul pays. C'est pour cela que cest onglet a été créé.

\hypertarget{uxe0-luxe9chelle-de-la-france}{%
\subsection{À l'échelle de la
France}\label{uxe0-luxe9chelle-de-la-france}}

\hypertarget{worldcoud}{%
\subsubsection{Worldcoud}\label{worldcoud}}

Le nuage de mots ou « wordcloud » en anglais est un outil de
visualisation qui permet au travers d'une image de percevoir très
rapidement quels sont les mots qui sont les plus fréquents au sein d'un
texte ou un corpus de texte. L'utilisateur peut en un clic sélectionner
une année et observer quels sont les prénoms qui ont été le plus
fréquemment donnés sur cette année, en lui laissant le choix de la
fréquence d'apparition du prénom ainsi que le nombre de prénoms qui
seront présents dans le wordcloud.

Wordcloud est un widget html. Cela signifie que votre wordcloud sera
sorti dans un HTMLformat. Nous avons décider de mettre en place un
bouton qui permet à l'utilisateur de l'exporter en tant que png image.

\hypertarget{courbe-du-pruxe9nom-au-fur-et-uxe0-mesure-des-annuxe9es}{%
\subsubsection{Courbe du prénom au fur et à mesure des
années}\label{courbe-du-pruxe9nom-au-fur-et-uxe0-mesure-des-annuxe9es}}

Cet onglet est sûrement le plus interactif avec l'utilisateur. Nous
sommes en effet amenés à écrire un prénom, et en fonction de celui-ci,
la courbe du nombre de bébés ayant reçu ce prénom s'affichera.

\hypertarget{uxe0-luxe9chelle-dune-maternituxe9}{%
\subsection{À l'échelle d'une
maternité}\label{uxe0-luxe9chelle-dune-maternituxe9}}

\hypertarget{regression-simple}{%
\subsubsection{Regression simple}\label{regression-simple}}

Nous allons etudier le poids de naissance des bébés.

Les variables sont :

\begin{itemize}
\item
  le poids de naissance du bébé (en grammes) (\textbf{PoidsBB})
\item
  l'âge de la mère (\textbf{AgedeleMere})
\item
  le poids de la mère en ??? (\textbf{PoidsMere})
\item
  la taille du bébé (en centimètre) (\textbf{TailleBB})
\item
  Sexe du bébé (fille = 0, garçon = 1) : transformation de la variable
  \textbf{Sexe} en varaible indicatrice
\end{itemize}

Nous considérons le modèle suivant :

\(Y_{PoidsBB} = \beta0 + \beta1 X1_{AgedelaMere} + \beta2 X2_{PoidsMere} + \beta3X3_{TailleBB} + \beta4X4_{Sexe} + \epsilon\)

Avant d'estimer les paramêtres, nous calculons la matrice de corrélation
et nous présenterons un diagramme de dispersion de toutes les pairs de
ces variables. Ceci permet de visualiser la relation entre la variable à
expliquer et chacune des variables explicatives et de juger de la
correlation entre les variables explicatives

\hypertarget{ruxe9gression-multiple}{%
\subsubsection{Régression multiple}\label{ruxe9gression-multiple}}

L'utilisateur a le choix de faire une régression multiple en choisissant
les différentes variables explicatives. Il n'a pas le choix concernant
la variable à explliquer ?

\hypertarget{travail-en-binuxf4me}{%
\section{Travail en binôme}\label{travail-en-binuxf4me}}

\hypertarget{utilisation-de-git}{%
\subsection{Utilisation de git}\label{utilisation-de-git}}

Afin de faciliter notre travail nous avons décidé d'utiliser git. En
effet, nous avons eu un cours en début de semestre et cela nous a semblé
pertinent de travailler avec cet outil. Nous avons pour ce projet
utilisé le terminal et non les boutons de R. Nous avons donc travaillé
en effectuant des git commit, des git push, des git pull et même un git
remote.

\hypertarget{ruxe9partition-des-tuxe2ches}{%
\subsection{Répartition des tâches}\label{ruxe9partition-des-tuxe2ches}}

Nous nous sommes réparties les tâches au fur et à mesure de l'avancée de
l'application. Nous avons tout d'abord décidé de la structure de
l'application ensemble, ainsi que des onglets et des sous-onglets. Nous
avons ensuite établi les grandes idées : les graphiques, où nous
voulions faire des cartes, quels packages nous pouvions utiliser qui
pouvaient rendre l'application intéressante etc. Après cela, nous nous
tenions au courant au fur et à mesure pour savoir qui faisait quoi. Bien
sûr, si l'une d'entre nous avait commencé une partie mais n'arrivait pas
à la finir ou ne trouvait pas ses erreurs, l'autre l'aidait ou prenait
le relai.

\hypertarget{notre-ressenti}{%
\subsection{Notre ressenti}\label{notre-ressenti}}

Nous avons beaucoup aimé travailler sur le thème des naissances. Avoir
différentes échelles à étudier nous a permis de ne pas nous répéter,
d'autant plus que nous avons utiliser plusieurs outils de visualisation
différents. Nous avons également beaucoup aimé travailler ensemble :
nous sommes un binôme qui fonctionne bien car nous avons la même méthode
de travail.

\hypertarget{difficultuxe9s-rencontruxe9es}{%
\subsection{Difficultés
rencontrées}\label{difficultuxe9s-rencontruxe9es}}

Nous avons chacune rencontré des difficultés en travaillant sur le
projet. Nous allons ici vous les présenter : * Utilisation du package
formattable, association avec le package DT qui marchait très bien dans
ma console mais impossible de le faire marcher dans l'application à
cause d'une erreur html * Base de données fertility, impossible de
sélectionner trop d'années car trop lourd * NA dans le choix des années
avec la carte, on ne sait pas comment l'enlever * Voulait faire un
bouton logout mais il fallait shiny server pro * Difficultés à publier
notre application : \textbf{Methode de resolution des problèmes} Pour
trouver les erreurs nous avons essayer de publier une application plus
basique et nous avons rajouter le code au fur et à mesure. Nous avons
mis du temps à réaliser cette étape car cela met du temps de verifier
étape par étape que notre application fonctionne en local et de publier.
aPR7 \textbf{Problème non resolu} - Problème racontrer au niveau de la
visualisation des bases de données au format DT sur nos grosses bases de
données ( prenom et taux de fecondité) - Problème rencontré lorsque que
l'on fait appel à un serveur externe type API qui nous permet de réalisé
une carte du taux de fertilité avec le package WDI

\textbf{Les hypothèses du problème} - Version shiny.io n'a pas les memes
versions de packages que celles qu'on utilise en local - Mauvais import
de base de données du a un mauvais trie (erreur manquante\ldots) -
Shiny.io protege son serveur et ne peut pas publier des applications qui
font appel à des serveurs externes

Ces difficultés nous ont permis de voir les limites de notre application
et de nous remettre sans cesse en question.

\hypertarget{conlusion}{%
\section{Conlusion}\label{conlusion}}

Pour conclure, cela a été un vrai plaisir de travailler sur ce projet.
Nous espérons que l'application vous plaira. Nous avons beaucoup appris
: gérer nos erreurs, gérer notre temps (le projet demandant une certaine
investigation et ayant une date limite de rendu), répartir les
tâches\ldots{}

\end{document}
